\documentclass[a4paper,10pt]{article}
%Avril 2016

\usepackage{amsmath, amssymb,amsfonts,  mathrsfs}
\usepackage{amsthm}%\usepackage[T1]{fontenc}
\usepackage{a4wide}
\usepackage[latin1]{inputenc}
\usepackage[francais]{babel}% donne de bonnes c\'esures  en francais
\usepackage{titlesec}%pour modifier le style des sections
\usepackage{float}%pour fixer les flottants
\usepackage{url}%pour ecrire une adresse d'un site web
\usepackage[all]{xy}%pour faire des diagrammes
\usepackage[dvips]{graphicx}
\usepackage{nopageno} % pour supprimer les numeros des pages
\usepackage{pdfsync}
\pagestyle{myheadings}
\markright{ENSIIE 1A Ann�e 2015/2016 \\ MPM2 Projet Math}
\newtheorem{definition}{D�finition}

\begin{document}
\begin{center}
\textbf{\huge{Rapport : Projet maths}}
\end{center}
\medskip

\section{Partie Math�matiques}

\subsection{D�riv�e de Gateau $J'(u,\delta u)$}

\renewcommand{\labelitemi}{$\bullet$}
\begin{itemize}
	\item D�terminer la d�riv�e de G�teau de la fonction $x(u)$.
	\medbreak

	On a :
	\begin{equation*}
		\left\{\begin{array}{l}
		\dot x(t) = Ax(t) + Bu(t)\\
		x(0) = x_0
		\end{array}
		\right.
	\end{equation*}
	\medbreak

	Et la formule de la d�riv�e de G�teau :
	\begin{equation*}
		f'(u,v) = \lim_{\lambda\to 0^+} \frac{f(u+\lambda) - f(u)}{\lambda}.
	\end{equation*}
	\medbreak
	
	On cherche :
	\begin{equation*}
		\left\{\begin{array}{l}
		\dot x'(u,\delta u)(t)\\
		x'(0)
		\end{array}
		\right.
	\end{equation*}
	
	Or $x'(0) = 0$ \\	
	Et :
	\begin{equation*}
	\begin{aligned}
		\dot x'(u,\delta u)(t) &= \lim_{\lambda\to 0^+} \frac{\dot x(u,\delta u)(t) - \dot x(u)(t)}{\lambda} \\
			&= \lim_{\lambda\to 0^+} \frac{Ax(u,\delta u)(t) + B(u,\delta u)(t) - [Ax(u)(t) + B(u)(t)]}{\lambda} \\
			&= \lim_{\lambda\to 0^+} \frac{A[x(u,\delta u)(t) - x(u)(t)] + B(u,\delta u)(t) -  B(u)(t)}{\lambda} \\
			&= Ax'(u,\delta u)(t) + \lim_{\lambda\to 0^+} \frac{B(u)(t) + B(\lambda\delta u)(t) - B(u)(t)}{\lambda} \\	
			&= Ax'(u,\delta u)(t) + B(\delta u)(t) \\				
	\end{aligned}		
	\end{equation*}	
	\medbreak
	
	On a donc la d�riv�e de G�teau de la fonction $x(u)$ :
	\begin{equation*}
		\left\{\begin{array}{l}
		\dot x'(u,\delta u)(t) = Ax'(u,\delta u)(t) + B(\delta u)(t)\\
		x'(0)= 0
		\end{array}
		\right.
	\end{equation*}
	
	\item En d�duire la d�riv�e de G�teau de $J$.
	\medbreak
	
	\begin{equation*}
\begin{aligned}
	J'(u,\delta u)(t) &= \lim_{\lambda\to 0^+} \frac{J(u,\lambda \delta u)(t) - J(u)(t)}{\lambda} \\
		&= \lim_{\lambda\to 0^+} \int_{0}^{T} \frac{\epsilon}{2} (u + \lambda \delta  u)� \, \mathrm{d}s + \frac{1}{2}x�(u+\lambda\delta u)(t) - \int_{0}^{T} \frac{\epsilon}{2} u�(t) \, \mathrm{d}s - \frac{1}{2} x�u(t)\\	
		&= \lim_{\lambda\to 0^+} \frac{1}{\lambda} [\int_{0}^{T} \frac{\epsilon}{2}(u�(s) + \lambda�\delta u�(s) \, \mathrm{d}s) -  \int_{0}^{T} \frac{\epsilon}{2} u�(s)]  \, \mathrm{d}s\\	
		&= \lim_{\lambda\to 0^+} \frac{1}{\lambda} [\frac{1}{2}x�(u+\delta u)(t) - \frac{1}{2} x�(u)(t)]\\
\end{aligned}		
\end{equation*}
On identifie par rapport � la d�riv�e de G�teau, d'o� :
\begin{equation*}
\begin{aligned}
	J'(u,\delta u)(t) &= \lim_{\lambda\to 0^+} \int_{0}^{T} \frac{\epsilon}{2} \frac{\lambda�\delta u�(s)+2\lambda u(s)\delta u(s)}{\lambda} \, \mathrm{d}s + [\frac{1}{2} x� u(t)]' \\
		&= \lim_{\lambda\to 0^+} \int_{0}^{T} \frac{\epsilon}{2} [\lambda\delta u�(s)+2 u(s)\delta u(s)] \, \mathrm{d}s + x'(u, \delta u)(t)x(t) \\
		&= \int_{0}^{T} \epsilon u(s)\delta u(s) \, \mathrm{d}s + x'(u, \delta u)(t)x(t) \\
\end{aligned}		
\end{equation*}
	
	\item D�terminer la fonction $g(t)$.
	\medbreak	
	
	En prenant la formule de la d�riv�e de G�teau appliqu�e pour la fonction $J$ on a :
	\begin{equation*}
		J'(u,\delta u)(t) = \lim_{\lambda\to 0^+} \frac{J(u,\delta u)(t) - J(u)(t)}{\lambda}
	\end{equation*}	
	
	D'apr�s la d�riv�e de Taylor-Young qui fait intervenir le gradient, on a : 
	\begin{equation*}
		J(u,`\lambda u)(t) = J(u)(t) + \lambda(\nabla J(u)(t), \delta u) + o(||\lambda \delta u||)
	\end{equation*}
	\begin{equation*}
		\Leftrightarrow J(u,`\lambda u)(t) - J(u)(t) = \lambda(\nabla J(u)(t), \delta u)
	\end{equation*}
	
	En rempla�ant on obtient :
	\begin{equation*}
	\begin{aligned}
		J'(u,\delta u)(t) &= \lim_{\lambda\to 0^+} \frac{\lambda(\nabla J(u)(t), \delta u)}{\lambda} \\
			&= (\nabla J(u)(t), \delta u) \\
			&= \int_{0}^{T} \nabla J(u)(s) \delta u(s) \, \mathrm{d}s
	\end{aligned}
	\end{equation*}	
	
	Par identification on voit donc que $g(t) = \nabla J$
	
\end{itemize}

\subsection{Calcul de $\nabla J$}

\subsection{R�solution num�rique}

\section{Partie Informatique}

\end{document}

%%% Local Variables:
%%% mode: latex
%%% TeX-master: t
%%% End:
